%% start of file `template.tex'.
%% Copyright 2006-2015 Xavier Danaux (xdanaux@gmail.com).
%
% This work may be distributed and/or modified under the
% conditions of the LaTeX Project Public License version 1.3c,
% available at http://www.latex-project.org/lppl/.

\documentclass[11pt,a4paper,sans]{moderncv}        % possible options include font size ('10pt', '11pt' and '12pt'), paper size ('a4paper', 'letterpaper', 'a5paper', 'legalpaper', 'executivepaper' and 'landscape') and font family ('sans' and 'roman')

% moderncv themes
\moderncvstyle{classic}                             % style options are 'casual' (default), 'classic', 'banking', 'oldstyle' and 'fancy'
\moderncvcolor{burgundy}                               % color options 'black', 'blue' (default), 'burgundy', 'green', 'grey', 'orange', 'purple' and 'red'
%\renewcommand{\familydefault}{\sfdefault}         % to set the default font; use '\sfdefault' for the default sans serif font, '\rmdefault' for the default roman one, or any tex font name
%\nopagenumbers{}                                  % uncomment to suppress automatic page numbering for CVs longer than one page

% character encoding
%\usepackage[utf8]{inputenc}                       % if you are not using xelatex ou lualatex, replace by the encoding you are using
%\usepackage{CJKutf8}                              % if you need to use CJK to typeset your resume in Chinese, Japanese or Korean
%\usepackage{hyperref}

% adjust the page margins
\usepackage[margin=0.75in,scale=1.0]{geometry}
%\setlength{\hintscolumnwidth}{3cm}                % if you want to change the width of the column with the dates
%\setlength{\makecvtitlenamewidth}{10cm}           % for the 'classic' style, if you want to force the width allocated to your name and avoid line breaks. be careful though, the length is normally calculated to avoid any overlap with your personal info; use this at your own typographical risks...

% personal data
\name{Vincent}{Zvikaramba}
%\title{Resumé title}                               % optional, remove / comment the line if not wanted
%\address{street and number}{postcode city}{country}% optional, remove / comment the line if not wanted; the "postcode city" and "country" arguments can be omitted or provided empty
% \phone[mobile]{}                   % optional, remove / comment the line if not wanted; the optional "type" of the phone can be "mobile" (default), "fixed" or "fax"
\email{zvikovincent@gmail.com}                             % optional, remove / comment the line if not wanted
%\homepage{www.johndoe.com}                         % optional, remove / comment the line if not wanted
\social[linkedin]{zvikaram}                        % optional, remove / comment the line if not wanted
%\social[twitter]{phhusson}                             % optional, remove / comment the line if not wanted
\social[github]{vince2678}                              % optional, remove / comment the line if not wanted
%\extrainfo{additional information}                 % optional, remove / comment the line if not wanted
%\photo[64pt][0.4pt]{picture}                       % optional, remove / comment the line if not wanted; '64pt' is the height the picture must be resized to, 0.4pt is the thickness of the frame around it (put it to 0pt for no frame) and 'picture' is the name of the picture file
%\quote{Some quote}                                 % optional, remove / comment the line if not wanted

% bibliography adjustements (only useful if you make citations in your resume, or print a list of publications using BibTeX)
%   to show numerical labels in the bibliography (default is to show no labels)
\makeatletter\renewcommand*{\bibliographyitemlabel}{\@biblabel{\arabic{enumiv}}}\makeatother
%   to redefine the bibliography heading string ("Publications")
%\renewcommand{\refname}{Articles}

% bibliography with mutiple entries
%\usepackage{multibib}
%\newcites{book,misc}{{Books},{Others}}
%----------------------------------------------------------------------------------
%            content
%----------------------------------------------------------------------------------

\begin{document}
%\begin{CJK*}{UTF8}{gbsn}                          % to typeset your resume in Chinese using CJK
%-----       resume       ---------------------------------------------------------
%	\vspace*{-15mm}
\makecvtitle
%	\vspace*{-10mm}

\section{Education}
\cventry{2021}{Honours Bachelor of Science}{University of Toronto}{}{}{Statistics, Computer Science}

\section{Skills}
\cvitem{Programming}{Typescript/Javascript, Python, HTML, CSS, Java, C/C++, PHP, SQL, Bash}%{}{}%{category 4}{XXX, YYY, ZZZ}
\cvitem{Tools}{Docker, Git, Node.js/Npm, Backbone.js, Django ORM, SQLAlchemy, GitLab CI, Jenkins, Celery, Nginx, Ansible}%{}{}%{category 5}{XXX, YYY, ZZZ}
\cvitem{Frameworks and libraries}{React.js, Socket.IO, Celery, JQuery, Java Collections, Guice, Gson, Guava, Lombok, JUnit, Spring Boot, SQLAlchemy}%{}{}%{category 6}{XXX, YYY, ZZZ}


\section{Experience}

\cventry{2022--Present}{Software Engineer}{}{TitanFile Inc}{}{
	\begin{itemize}
		\item Reviewed other developers' code as part of code reviews and participated in daily stand-up meetings summarizing daily progress
		\item Worked with the Infrastructure Manager as part of the DevOps team to improve deployment process and migrate to using containerization using Docker
		\item Reimplemented parts of Docusign integration to use more modern JSON instead of XML webhooks
		\item Improved display of embedded images in emails sent via application's Outlook integration plugin
		\item Identified and optimized memory usage issue for PDF watermarking process
		\item Worked as a project leader on an internal hackathon to develop an integration with ChatGPT to leverage the power of AI to improve the user experience
		\item Developed Microsoft 365 integration using Microsoft's WOPI specification from scratch to enable viewing and editing documents within the application, implementing new code in TypeScript using React.js framework
		\item Rewrote frequently used views from using untyped Javascript and JQuery to TypeScript/ES6 using React.js and TSX, helping identify and reduce uncaught bugs and increasing reusability and readability
		\item Helped onboard and train new developers, particularly on the tech stack and DevOps practices
	\end{itemize}
}

\cventry{2016--2017}{Teaching Assistant}{Software Tools and Systems Programming}{University of Toronto}{}{
	\begin{itemize}
		\item Instructed students in tutorials and helped them troubleshoot and identify bugs in programming assignments
		\item Graded programming assignments and provided feedback for improvement
		\item Worked as part of course TAs to proctor tests and mark test papers
	\end{itemize}
}
\newpage
%\vspace*{-3mm}
\section{Projects}
\cventry{2022--Present}{Client Management System}{South-Asian Women's Rights Organization}{Toronto}{}{
	\begin{itemize}
		\item Worked as the lead developer of a two-person team to (re)write the client management system that was initially written in Flask and hand-written SQL code
		\item Rewrote the CMS using Django as the backend, MySQL as the database and React.js and TS/ES6 for the frontend
		\item Designed REST backend using and Django to handle passing client data between the SQL database and frontend
	\end{itemize}
}
\cventry{2022--Present}{Website}{Promatec Solutions}{South Africa}{}{
	\begin{itemize}
        \item Collaborated with another developer to build a website and e-mail solution for Promatec
        \item Used docker for containerisation and easy deployment of services
        \item Deployed and leveraged Gerrit Code Review for collaboration and code review
        \item Wrote a python script hooking into the CloudFlare API for managing DNS records required or used in application containers
        \item Used Node.js as web server and as the backend for APIs exposed to the frontend
        \item Used React.js for UI component design and reuse
	\end{itemize}
}
\cventry{2015--2019}{\href{https://github.com/Galaxy-MSM8916}{Android Custom ROMs}}{}{}{}{
	\begin{itemize}
        \item Analysing Android system logs to debug  builds and create SELinux security policies
		\item Maintained support for Samsung devices originally running Android versions 4.0 through 6.0, allowing them to run newer Android
        versions up to and including Android 10 (Q)
		\item Maintained kernel code provided by the phone and board OEMs (Samsung, Qualcomm),
backporting upstream security fixes, new kernel features and driver updates to legacy code
		\item Developed user-space android system tools and interposing libraries to provide forward compatibility for newer versions of Android to
proprietary OEM libraries and programs
        \item Went through API documentation to create shim or wrapper libraries to maintain comatibility with proprietary libraries from older versions of Android
		\item Customised and created Makefiles for use with Android's various build systems (Make, Ninja)
		\item Deployed Jenkins for automated builds with Gerrit for code review and collaboration with GIT as the VCS backend
	\end{itemize}
}

\end{document}


%% end of file `template.tex'.
