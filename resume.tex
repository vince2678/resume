%% start of file `template.tex'.
%% Copyright 2006-2015 Xavier Danaux (xdanaux@gmail.com).
%
% This work may be distributed and/or modified under the
% conditions of the LaTeX Project Public License version 1.3c,
% available at http://www.latex-project.org/lppl/.

\documentclass[11pt,a4paper,sans]{moderncv}        % possible options include font size ('10pt', '11pt' and '12pt'), paper size ('a4paper', 'letterpaper', 'a5paper', 'legalpaper', 'executivepaper' and 'landscape') and font family ('sans' and 'roman')

% moderncv themes
\moderncvstyle{classic}                             % style options are 'casual' (default), 'classic', 'banking', 'oldstyle' and 'fancy'
\moderncvcolor{burgundy}                               % color options 'black', 'blue' (default), 'burgundy', 'green', 'grey', 'orange', 'purple' and 'red'
%\renewcommand{\familydefault}{\sfdefault}         % to set the default font; use '\sfdefault' for the default sans serif font, '\rmdefault' for the default roman one, or any tex font name
%\nopagenumbers{}                                  % uncomment to suppress automatic page numbering for CVs longer than one page

% character encoding
%\usepackage[utf8]{inputenc}                       % if you are not using xelatex ou lualatex, replace by the encoding you are using
%\usepackage{CJKutf8}                              % if you need to use CJK to typeset your resume in Chinese, Japanese or Korean
%\usepackage{hyperref}

% adjust the page margins
\usepackage[margin=0.75in,scale=1.0]{geometry}
%\setlength{\hintscolumnwidth}{3cm}                % if you want to change the width of the column with the dates
%\setlength{\makecvtitlenamewidth}{10cm}           % for the 'classic' style, if you want to force the width allocated to your name and avoid line breaks. be careful though, the length is normally calculated to avoid any overlap with your personal info; use this at your own typographical risks...

% personal data
\name{Vincent}{Zvikaramba}
%\title{Resumé title}                               % optional, remove / comment the line if not wanted
%\address{street and number}{postcode city}{country}% optional, remove / comment the line if not wanted; the "postcode city" and "country" arguments can be omitted or provided empty
\phone[mobile]{+1~647~974~5597}                   % optional, remove / comment the line if not wanted; the optional "type" of the phone can be "mobile" (default), "fixed" or "fax"
\email{zvikovincent@gmail.com}                             % optional, remove / comment the line if not wanted
%\homepage{www.johndoe.com}                         % optional, remove / comment the line if not wanted
\social[linkedin]{zvikaram}                        % optional, remove / comment the line if not wanted
%\social[twitter]{phhusson}                             % optional, remove / comment the line if not wanted
\social[github]{vince2678}                              % optional, remove / comment the line if not wanted
%\extrainfo{additional information}                 % optional, remove / comment the line if not wanted
%\photo[64pt][0.4pt]{picture}                       % optional, remove / comment the line if not wanted; '64pt' is the height the picture must be resized to, 0.4pt is the thickness of the frame around it (put it to 0pt for no frame) and 'picture' is the name of the picture file
%\quote{Some quote}                                 % optional, remove / comment the line if not wanted

% bibliography adjustements (only useful if you make citations in your resume, or print a list of publications using BibTeX)
%   to show numerical labels in the bibliography (default is to show no labels)
\makeatletter\renewcommand*{\bibliographyitemlabel}{\@biblabel{\arabic{enumiv}}}\makeatother
%   to redefine the bibliography heading string ("Publications")
%\renewcommand{\refname}{Articles}

% bibliography with mutiple entries
%\usepackage{multibib}
%\newcites{book,misc}{{Books},{Others}}
%----------------------------------------------------------------------------------
%            content
%----------------------------------------------------------------------------------

\begin{document}
%\begin{CJK*}{UTF8}{gbsn}                          % to typeset your resume in Chinese using CJK
%-----       resume       ---------------------------------------------------------
%	\vspace*{-15mm}
\makecvtitle
%	\vspace*{-10mm}

\section{Education}
\cventry{2021}{Honours Bachelor of Science}{University of Toronto}{}{}{Statistics, Computer Science}

\section{Skills}
\cvitem{Programming}{Typescript/Javascript, Python, HTML, CSS, Java, C/C++, PHP, SQL, (POSIX) Shell}%{}{}%{category 4}{XXX, YYY, ZZZ}
\cvitem{Tools}{Docker, Git, Node.js/Npm, Backbone.js, Underscore.js, Django ORM, GitLab, Celery, Nginx, Ansible}%{}{}%{category 5}{XXX, YYY, ZZZ}
\cvitem{Frameworks and libraries}{React.js, Socket.IO, Celery, JQuery, Java Collections, Guice, Gson, Guava, Lombok, JUnit, Spring Boot, SQLAlchemy}%{}{}%{category 6}{XXX, YYY, ZZZ}

\section{Interests}
\cvitem{Electronics}{I enjoy tinkering with hardware at every level. Particularly enjoyed working with FPGAs and Verilog in university for programmatic circuit design.}
\cvitem{Reading}{I come from a family of avid readers; I'm usually reading research articles, documentation or spec sheets for miscellaneous electronic components.}
\cvitem{Music}{I love listening to and playing music on guitar.}

\section{Experience}

\cventry{2022--Present}{Software Engineer}{}{TitanFile Inc}{}{
	\begin{itemize}
		\item Reviewed other developers' code ad part of code reviews and participated in daily stand-up meetings summarizing daily progress
		\item Worked with senior developers as part of DevOps team to improve performance and security of tools/infrastructure used for development
		\item Improved display of embedded images in emails sent via application's Outlook integration plugin
		\item Identified and optimized memory usage issue for pdf watermarking process
		\item Developed Microsoft 365 integration to enable editing documents within the web application, implementing new code in TypeScript and using React.js
	\end{itemize}
}

\cventry{2016--2017}{Teaching Assistant}{Software Tools and Systems Programming}{University of Toronto}{}{
	\begin{itemize}
		\item Instructed students in tutorials and helped them troubleshoot and identify bugs in programming assignments
		\item Graded programming assignments and provided feedback for improvement
		\item Worked as part of a team of TAs to proctor tests and mark test papers
	\end{itemize}
}

%\vspace*{-3mm}
\section{Projects}
\cventry{2022--Present}{Client Management System}{South-Asian Women's Rights Organization}{Toronto}{}{
	\begin{itemize}
		\item Worked as part of a team to build a client management system for SAWRO's
		client database
		\item Used Python and Flask to write the web application
		\item Used an ORM (SQLAlchemy and Django) to interact with the SQL database
		\item Rewrote CMS in Javascript and Python3 using Django ORM and MySQL for the backend and React.js and TS/ES6 for the frontend
	\end{itemize}
}
\cventry{2022--Present}{Website}{Promatec Solutions}{South Africa}{}{
	\begin{itemize}
        \item Collaborated with another developer to build a website and e-mail solution for Promatec
        \item Used docker for containerisation and easy deployment of services
        \item Deployed and leveraged Gerrit Code Review for collaboration and code review
        \item Wrote a python script hooking into the CloudFlare API for managing DNS records required or used in application containers
        \item Used Node.js as web server and as the backend for APIs exposed to the frontend
        \item Used React.js for UI component design and reuse
	\end{itemize}
}
\cventry{2015--2019}{\href{https://github.com/Galaxy-MSM8916}{Android Custom ROMs}}{}{}{}{
	\begin{itemize}
		\item Maintained support for Samsung devices originally running Android versions 4, 5, 6.0,
		\item Maintained legacy kernel code and
backported new kernel code
		\item Developed tools and user-space interposer libraries for forward compatibility of
proprietary libraries and programs with future android releases
		\item Maintained Makefiles for use in the build system
		\item Deployed Gerrit for code review and collaboration and Jenkins for builds and continuous integration
	\end{itemize}
}

\end{document}


%% end of file `template.tex'.
